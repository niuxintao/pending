\documentclass[format=acmsmall, review=false, screen=true]{acmart}

\usepackage{booktabs} % For formal tables

\usepackage[ruled]{algorithm2e} % For algorithms
\renewcommand{\algorithmcfname}{ALGORITHM}
\SetAlFnt{\small}
\SetAlCapFnt{\small}
\SetAlCapNameFnt{\small}
\SetAlCapHSkip{0pt}
\IncMargin{-\parindent}


% Metadata Information
\acmJournal{TWEB}
\acmVolume{9}
\acmNumber{4}
\acmArticle{39}
\acmYear{2010}
\acmMonth{3}
\copyrightyear{2009}
%\acmArticleSeq{9}

% Copyright
%\setcopyright{acmcopyright}
\setcopyright{acmlicensed}
%\setcopyright{rightsretained}
%\setcopyright{usgov}
%\setcopyright{usgovmixed}
%\setcopyright{cagov}
%\setcopyright{cagovmixed}

% DOI
\acmDOI{0000001.0000001}

% Paper history
\received{February 2007}
\received[revised]{March 2009}
\received[accepted]{June 2009}


% Document starts
\begin{document}
% Title portion. Note the short title for running heads
\title[]{The pending schema of Combinatorial Testing}

\author{Xintao Niu}
\orcid{0000-0001-5786-0894}
\affiliation{%
  \institution{State Key Laboratory for Novel Software Technology, Nanjing University}
  \streetaddress{163 Xianlin Road, Qixia District}
  \city{Nanjing}
  \state{Jiangsu}
  \postcode{210023}
  \country{China}}
\email{niuxintao@gmail.com}
\author{Huayao Wu}
\affiliation{%
  \institution{State Key Laboratory for Novel Software Technology, Nanjing University}
  \streetaddress{163 Xianlin Road, Qixia District}
  \city{Nanjing}
  \state{Jiangsu}
  \postcode{210023}
  \country{China}}
\email{hywu@outlook.com}
\author{Changhai Nie}
\affiliation{%
    \institution{State Key Laboratory for Novel Software Technology, Nanjing University}
  \streetaddress{163 Xianlin Road, Qixia District}
  \city{Nanjing}
  \state{Jiangsu}
  \postcode{210023}
  \country{China}
}
\email{changhainie@nju.edu.cn}
\author{Yu Lei}
\affiliation{%
 \institution{Department of Computer Science and Engineering, The University of Texas at Arlington}
 %\streetaddress{Rono-Hills}
 \city{Arlington}
 \state{Texas}
 \country{USA}}
\email{ylei@cse.uta.edu}
\author{Xiaoyin Wang}
\orcid{0000-0002-9079-5534}
\affiliation{%
  \institution{Department of Computer Science, The University of Texas at San Antonio}
  %\streetaddress{30 Shuangqing Rd}
  \city{San Antonio}
  \state{Texas}
  \country{USA}
}
\email{Xiaoyin.Wang@utsa.edu}
\author{Fei-Ching Kuo}
\affiliation{%
  \institution{Swinburne University of Technology}
  \city{Melbourne}
  \country{Australia}}
\email{dkuo@swin.edu.au}



\begin{abstract}
Combinatorial testing (CT) aims to detect the failures which are triggered by the interactions of various factors that can influence the behaviour of the system, such as input parameters, and configuration options. Many studies in CT focus on designing an elaborate test suite (called covering array) to reveal such failures. Although covering array can assist testers to systemically check each possible factor interaction, however, it provides weak support to locate the failure-inducing interactions, i.e., the Minimal Failure-causing Schemas (MFS). Recently some elementary researches are proposed to handle the MFS identification problem. However, we argue that many of them are still incomplete in terms of the existence of schemas that cannot be determined to be faulty or not yet. These cannot-be-determined schemas, i.e., the pending schemas, would be hidden dangers to the software under testing. Hence, it is important to obtain these pending schemas for these incomplete MFS identification approaches. In this paper, we proposed several propositions to formulate the set of pending schemas and give three equivalent formulas to obtain them, based on which we reduce the complexity of obtaining pending schemas from $O(2^{n})$ to $O(\tau^{|\mathcal{C}(T_{fail})^{t}|} \times \tau^{|T_{pass}^{t\bigtriangleup}|})$, where $n$ is the number of factors in the software, while $|\mathcal{C}(T_{fail})^{t}|$ and $|T_{pass}^{t\bigtriangleup}|$ are two relatively small numbers and independent on the number of $n$. We conduct a series empirical studies on some real software systems with various number of parameters and values. Our results shows that the incompleteness is very common in the covering arrays and  MFS identification approaches. We also observed that the third proposed formula is the most efficient when compared others in most cases.
\end{abstract}


%
% The code below should be generated by the tool at
% http://dl.acm.org/ccs.cfm
% Please copy and paste the code instead of the example below.
%
\begin{CCSXML}
<ccs2012>
<concept>
<concept_id>10011102.10011103</concept_id>
<concept_desc>Software defect analysis~Software testing and debugging</concept_desc>
<concept_significance>500</concept_significance>
</concept>
</ccs2012>
\end{CCSXML}

\ccsdesc[500]{Software defect analysis~Software testing and debugging}

%
% End generated code
%


\keywords{Pending Schema, Minimal Failure-causing Schema, Combinatorial Testing, Software Testing}




\maketitle

% The default list of authors is too long for headers.
\renewcommand{\shortauthors}{Xintao Niu et al.}

\input{samplebody-journals}


\end{document}
